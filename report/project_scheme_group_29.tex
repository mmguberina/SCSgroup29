\documentclass{article}
\usepackage{amsmath}
\usepackage{amssymb}
\usepackage{amsfonts}
\usepackage[T1]{fontenc}
\usepackage{bm}
\usepackage{array}
\usepackage{graphicx}
\usepackage[utf8]{inputenc}

\title{SCS group 29 Project Scheme}
\date{2020-11-27}
\author{Marko Guberina, Christian Josefsson, Anders Segerlund}

\begin{document}
\pagenumbering{gobble}
\maketitle
\newpage
\pagenumbering{arabic}

\section{Introduction}
The goal of the project is to simulate a simplified model 
of a warehouse with many robot workers in order to test whether 
the robots can produce emergent swarm-like behaviour which yields
efficient product delivery.

Two different models are developed simultaneously:

\begin{enumerate}
		\item a discrete model
		\item a continuous model.
\end{enumerate}

This is done to see whether similar behaviour emerges
in both cases as today many localization systems for robots are based 
on a discrete grid, but discretizing the model might effect the
resulting behaviour. For this reason we think both versions should be implemented.

Furthermore, in order to check whether this model of robot control
is efficient, it is compared against results obtained from a
random walk model and from a simple deterministic task assignmnet scheme.

\section{General notes the models}
In the discrete model, the warehouse is a 2-dimensional grid where
all objects are represented as vertexes.
In the continuous model, robots and items are represented as circular
particles.
Delivery of picked-up items is performed in a determistic manner ---
the robots go straight to the dispatching station. 
The items to be picked up appear randomly on the warehouse floor.
The robots search for items in a fashion similar to that of active Brownian
motion (random walks in the discrete case), 
but in such a manner that they are repelled other robots and are attracted 
to the items. 
The robots movement is effected by what is in their visibility radius which is 
modelled a circle centered at the robot's position and a random parameter.

The models used for benchmarking fundamentally differ only in robot movement
while searching for items. In the random benchmark the robot's movements
are Brownian motion/random walks. In the determistic assignmnet model
the robots are assigned item(s) to be picked up which they then go and pick up
using the shortest route. They are assigned based on their 
position relative to the robot's positions.

The models contain a lot of parameters whose correct values are hard to determine,
but whose precision is needed to achieve highest possible item picking efficiency.
For this reason, all models will be optimised with the genetic algorithm.

\section{Current state of project progress}
Right now both models are nearly implemented (require some debugging before
they work as expected) along with the random benchmark. 
Right now we are working on what kind of function should be used
as the attracting and repelling force to achieve the imagined robot movement.

The implementation can be observed in the Github repository 
\newline
https://github.com/mmguberina/SCSgroup29/
\newline
but since we need to keep the repository private, in order to see it please 
send us your Github account name or email so that we can add you.

We have not yet started writing the report.

\section{Desired additions to the model}
To make the models more realistic, we would like to add further complications
whose implementation depends on the amount of time we have.
Those are:

\begin{enumerate}
		\item prioritizing the items (by time and/or by some other criterion)
		\item adding bariers (walls) to the warehouse floor
		\item having more delivery stations
		\item making the simulation more realistic by tuning the model
\end{enumerate}

\end{document}
