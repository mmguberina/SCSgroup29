\section{Introduction}
This project explores the usage of random search algorithms in swarm robotics.
We first introduce the concept of swarm robotics. 
Many robotics applications call for usage of multiple robots, search being one of them, as it
clearly it is easier to find desired targets if there are multiple agents performing the search operation.
The term swarm robotics denotes the case when there are many robotic agents which when working together
exhibit collective (group) behaviour.
These robots are typically very simple and their operation is often inspired by animals such 
as ants or bees.
The benefits of using a large number of simple robots rather than a few more capable and complex robots
are the following:
\begin{itemize}
		\item price -- such robots are very cheap to produce as they require just the basic of-the-shelf components
		\item robustness -- failure of a single robot has a small impact on the overall result
		\item scalability -- swarms are designed to handle a very large number of agents and new agents
				can very easily be introduced as the swarm is typically controled in a distributed fashion
		\item equally complex results 
				-- even through the robots themselves are simple, they can produce complex
				emergent behaviours and are thus very adaptable
\end{itemize}
Some of the above points need to be elaborated further. 
Robustness is achieved by having all robots identical in both form and function which
makes them mutually interchangable. 
Having indentical robots also makes it easy to simply add more robots.
However, additional care in algorithm design is required to avoid diminishing returns
when adding new robots. Of course, at some point cluttering will prevent robots from moving,
but we aim to achieve the full potential of the swarm.
Achieving equally complex results is the key feature which needs to be delivered. 
In essence, this means that we should view the swarm as a single ``distributed robot''
which makes intelligent decisions to obtain the desired result. In other words,
the swarm needs to feature collective emergent behaviour which makes it better
than the sum of its parts, just like ants or bees.

Now let us address the problem we wish to solve:
achieving efficient autonomous search in unknown and/or changing environments.
How a search problem is solved critically depends on a few factors:
existance of a map and the ability of robots to localize themselves in the map.
Clearly, if the robots know where they are and where they could go, classic search
algorithms like A* would deliver provably optimal results.
However, many environments are not (accuratelly) mapped, like forests or ocean floors,
or they change over time. Furthermore, to localize robots must use GPS or
have known landmarks from which they can estimate their position.
This is so because odometry (starting from a known position and 
then integrating velocity (which is locally known)) very quickly accumulates errors which makes it 
unusable for all but short paths.
In this project we deal with unknown GPS denied environments.

Before proceeding to our solution, we need to discuss another important problem:
coordination. In order to use deterministic coordination algorithms, the robots
need to communicate. Given the environments we are interested in, robots would need to
endowed with powerful radio antenas to communicate with a global central node and
that would overall be very impractical. 
Thus the only communication option in this case would be local communication between neighbouring robots.
In the current implementation of the project this is not used, but there are situations in which
it would be useful, for example to signal presence of multiple found targets.
This is left as future work.



The above leads us to the following solution: using random walks as search algorithms.
This choise is in line with our design desires because it takes very little computational
power to implement these algorithms and that enables us to have (very) small and low powered robots.
One type of random walks are Levy walks. The idea is inspired by nature as 
this type of walk can be can be observed in hunting paterns of many different types of animals.
More importantly, we can mathematically show that this leads to higher area coverage efficiency 
in an obstacle free environment when compared to other types of random walks.
However, the situation is much more complicated in environments with obstacles. Namely, because
the obstacles prevent the free movement of robots, the possible and the resulting walking patterns will 
be different. 
Furthermore, the robots must not colide with obstacles or other robots. Even with more lax
collision requirements, they should at least
avoid them to prevent getting stuck. To combat this problem, artificial potential fields are employed.
In essence this just means that robots are repelled from obstacles each other by some amount.
Another benefit of artificial potential fields in this case is that they help with diffusion of
robots accross the environment. Given the role of artificial potential fields in this project, 
they are manually defined to satisfy the collision avoidance requirement.
These new complications introduced by obstacles lead us to our research questions:

\begin{enumerate}
\item How do different types of random search algorihms perform in different types of environments?
\item Will switching and dynamic tuning of the random search algorihms produce better results?
\end{enumerate}

The rest of this project report is organized as follows: first we mathematically define 
the random walks. We then introduce the environments we are interested in and 
explain our usage of artificial potential fields.
Next we present the derived robot control algorithms show their performance accross different 
environments. Finally we provide a discussion of the results and the summary of our findings.



There are a multitude of scenarios which consist of a search through unknown environments; some examples are: search and rescue in a burning building, searching for wildfires in a forest, trash collection (urban and in nature), mapping indoor environments and search for pollution. \TODO{Find references}

%\begin{itemize}
%\item Finding wildfires
%\item Searching for people
%\item Finding and picking up trash
%\item Mapping indoor environments
%\item Data collection in complex environments
%  \begin{itemize}
%  \item Fish population in oceans
%  \item Pollution
%  \end{itemize} 
%\end{itemize} 

\TODO{Paragraphs briefly explaining brownian motion, active brownian motion and Levy flights. Not very math-y, yet; we'll go into the proper theory in Methods. Just ``Brownian motion is a kind of random motion, used to model the random motion of particles, and was developed by Einstein back in the day'' etc.}

